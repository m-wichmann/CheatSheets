\documentclass[10pt,a4paper,landscape,DIV=calc,english]{scrartcl}

\usepackage[pdftex,
            landscape,
            top=1cm, bottom=1cm, left=1cm, right=1cm
           ]{geometry}

\usepackage[utf8]{inputenc}
\usepackage[T1]{fontenc}
\usepackage[english]{babel}

\usepackage{setspace}
\usepackage{microtype}
\usepackage{lmodern}

\usepackage[pdftex,
            bookmarks, bookmarksopen, bookmarksopenlevel=1, bookmarksnumbered=true,
            pdfpagemode={UseNone},
            pdfpagelayout={SinglePage},
            plainpages=false,
            pdfkeywords={},
            pdfsubject={},
            pdftitle={},
            pdfauthor={}
           ]{hyperref}

\usepackage{tabu}
\usepackage{booktabs}
\usepackage[table]{xcolor}
\usepackage{colortbl}
\usepackage{multirow}
\usepackage{multicol}

\pagestyle{empty}
\hyphenation{}

%\definecolor{blue_light}{rgb}{0.6, 0.9, 0.95}
%\definecolor{blue_dark}{rgb}{0.2, 0.4, 0.5}

\definecolor{cs_table_1}{rgb}{0.6, 0.8, 0.7}
\definecolor{cs_table_2}{rgb}{1.0, 1.0, 1.0}
\definecolor{cs_header}{rgb}{0.1, 0.3, 0.3}
\definecolor{cs_header_font}{rgb}{1.0, 1.0, 1.0}

%%%%%%%%%%%%%%%%%%%%%%%%%%%%%%%%
% Cheat sheet stuff

% Useage:
%   \begin{cschapter}[0.1]{Arithmetic}
%       \csitemhead{\textbf{Built-in Exceptions}}
%       \csitem{BaseException}{Exception}
%   \end{cschapter}


\newenvironment {cschapter}[2][0.5]
                {
                    \taburowcolors[]2{cs_table_1 .. cs_table_2}
                    \begin{tabu}{ X[#1, l] X[1 - #1, l] }

                    \multicolumn{2}{c}{
                        \rule{0pt}{0.45cm}
                        \cellcolor{cs_header}
                        \textcolor{cs_header_font}{\textbf{#2}}
                    }
                    \\[0.1cm]
                }
                {
                    \end{tabu}
                    \vspace{0.4cm}
                }

\newcommand{\csitem}[2]{{#1} & {#2}\\}

% TODO: auto adjust columnwidth
\newcommand{\csitemhead}[1]{\multicolumn{2}{p{0.9525\linewidth}}{#1}\\}

\newcommand{\cstitle}[1]{
                            \begin{center}
                            \Large{\textbf{#1}}\\
                            \end{center}
                        }

\newcommand{\code}[1]{\texttt{#1}}

% /Cheat sheet stuff
%%%%%%%%%%%%%%%%%%%%%%%%%%%%%%%%











































\begin{document}
\raggedright
\begin{multicols}{3}

    \cstitle{Python Cheat Sheet}

    %%%%%%%%%%%%%%%%%%%%%%%%%%%%%%%%%%%%%%%%
    \begin{cschapter}{Source File Basic}
        \csitemhead{\code{\#!/usr/bin/env python3}}
        \csitemhead{\code{\# -*- coding: utf-8 -*-}}
        \csitemhead{\code{if \_\_name\_\_ == "\_\_main\_\_":}}
    \end{cschapter}

    %%%%%%%%%%%%%%%%%%%%%%%%%%%%%%%%%%%%%%%%
    \begin{cschapter}{Definitions}
        \csitem{\code{class Clazz(object):}}{New Class inherting from 'object'.}
        \csitem{\code{def method(self):}}{New instance method.}
        \csitem{\code{def method(cls):}}{New class method.}
        \csitem{\code{def function():}}{New function.}
        \csitem{\code{def f(i=42):}}{Default value for named argument.}
    \end{cschapter}

    %%%%%%%%%%%%%%%%%%%%%%%%%%%%%%%%%%%%%%%%
    \begin{cschapter}{Class Special Methods}
        \csitem{\code{\_\_init\_\_(self,args)}}{Constructor}
        \csitem{\code{\_\_del\_\_(self)}}{Destructor}
        \csitem{\code{\_\_repr\_\_(self)}}{Offical string representation}
        \csitem{\code{\_\_str\_\_(self)}}{Nice string representation}
        \csitem{\code{\_\_cmp\_\_(self,other)}}{Compare 'self' to 'other'}
        \csitem{\code{\_\_index\_\_(self)}}{Index}
        \csitem{\code{\_\_hash\_\_(self)}}{Hash}
    \end{cschapter}

    %%%%%%%%%%%%%%%%%%%%%%%%%%%%%%%%%%%%%%%%
    \begin{cschapter}{\href{http://docs.python.org/3/library/stdtypes.html}{Slicing / List / Tuple}}
        \csitemhead{\code{a = [0,1,2,3,4,5], b = "hello", c = (1,2,3)}}
        \csitem{\code{a[3] --> 3}}{Indexing.}
        \csitem{\code{a[-2] --> 4}}{Backwards indexing.}
        \csitem{\code{a[1:] --> [1,2,3,4,5]}}{Slice from 1 to end.}
        \csitem{\code{a[:5] --> [0,1,2,3,4]}}{Slice from Begin to 5.}
        \csitem{\code{a[:-2] --> [0,1,2,3]}}{Slice to second last.}
        \csitem{\code{a[1:3] --> [1,2]}}{Slice from 1 to 3.}
        \csitem{\code{a[1:-1] --> [1,2,3,4]}}{Slice from 1 to last.}
        \csitem{\code{a[::2] --> [0,2,4]}}{Slice with step 2.}
        \csitem{\code{a[::-2] --> [5,3,1]}}{Backwards slice with step.}
        \csitem{\code{a[b=a[:]]}}{Shallow copy of a}
        \csitem{\code{4 in a --> True}}{Check for item in list.}
        \csitemhead{\textbf{Same methods for lists, strings, tuples}}
        \csitem{\code{"hell" in b --> True}}{Substring in string.}
        \csitem{\code{c[1:3] --> (2,3)}}{Slicing of tuples.}
    \end{cschapter}

    %%%%%%%%%%%%%%%%%%%%%%%%%%%%%%%%%%%%%%%%
    \begin{cschapter}{\href{http://docs.python.org/3/library/stdtypes.html}{Dictionary}}
        \csitemhead{\code{a = \{"a": 42, 3: "y"\}}}
        \csitem{\code{a["a"] --> 42}}{}
        \csitem{\code{a[3] = "z"}}{}
        \csitem{\code{a[3] --> "z"}}{}
        \csitem{\code{a.keys() --> ["a", 3]}}{}
        \csitem{\code{a.values() --> [42,"z"]}}{}
        \csitem{\code{del a[3]}}{}
        \csitem{\code{a --> \{"a": 42\}}}{}
    \end{cschapter}

    %%%%%%%%%%%%%%%%%%%%%%%%%%%%%%%%%%%%%%%%
    \begin{cschapter}{\href{http://docs.python.org/2/tutorial/controlflow.html}{Control flow}}
        \csitem{Conditional}{Loop (for)}
        \csitem{\code{if x < 0:}}{\code{for i in range(12):}}
        \csitem{\code{{\space\space\space\space}print("-")}}{\code{{\space\space\space\space}print(i)}}
        \csitem{\code{elif x == 0:}}{\code{for c in "string":}}
        \csitem{\code{{\space\space\space\space}print("0")}}{\code{{\space\space\space\space}print(c)}}
        \csitem{\code{else:}}{\code{while True:}}
        \csitem{\code{{\space\space\space\space}print("+")}}{\code{{\space\space\space\space}pass \# do nothing}}
        \csitemhead{Boolean Condition: and, or, not, is, in}
        \csitem{break}{Jump out of loop.}
        \csitem{continue}{Jump back to top of loop.}
        \csitem{range(start[,end[,step]])}{Get new list e.g. for loop.}
    \end{cschapter}

    %%%%%%%%%%%%%%%%%%%%%%%%%%%%%%%%%%%%%%%%
    \begin{multicols}{2}
        \begin{cschapter}[0.1]{Arithmetic}
            \csitem{+}{Addition}
            \csitem{-}{Subtraction}
            \csitem{*}{Multiplication}
            \csitem{/}{Division}
            \csitem{//}{Floored qoutient}
            \csitem{\%}{Remainder}
            \csitem{-x}{Negated}
            \csitem{**}{Power}
        \end{cschapter}
        \begin{cschapter}[0.1]{Bit logic}
            \csitem{|}{or}
            \csitem{\^{}}{xor}
            \csitem{\&}{and}
            \csitem{{<}{<}}{left shift}
            \csitem{{>}{>}}{right shift}
            \csitem{\~{}}{invert}
        \end{cschapter}
    \end{multicols}

    %%%%%%%%%%%%%%%%%%%%%%%%%%%%%%%%%%%%%%%%
    \begin{cschapter}{Exceptions}
        \csitem{\code{try:}}{Start protecting area.}
        \csitem{\code{{\space\space\space\space}fd = open(f, 'r')}}{Raise a new Exception.}
        \csitem{\code{except Exception:}}{Catch IOError.}
        \csitem{\code{{\space\space\space\space}print("error")}}{Exception handling.}
        \csitem{\code{else:}}{If no Exception raised...}
        \csitem{\code{{\space\space\space\space}fd.close()}}{...do something.}
        \csitemhead{\textbf{Built-in Exceptions:} BaseException, Exception, IndexError, KeyError, KeyboardInterrupt, OSError}
    \end{cschapter}

    %%%%%%%%%%%%%%%%%%%%%%%%%%%%%%%%%%%%%%%%
    \begin{cschapter}{\href{http://www.python.org/dev/peps/pep-0008/}{Codestyle (PEP8)}}
        \csitem{Indentation}{4 Spaces}
        \csitem{Maximum Line Length}{79 Characters}
        \csitem{Encoding}{UTF-8}
        \csitem{Doctstring}{"""Some docstring"""}
        \csitemhead{\textbf{Visibility Modifier}}
        \csitem{'protected'}{\code{\_* --> \_some\_name}}
        \csitem{'private'}{\code{\_\_* --> \_\_some\_name}}
        \csitemhead{\textbf{Naming Conventions}}
        \csitem{Package/Module}{Lowercase}
        \csitem{Class}{First letter uppercase, CamelCase}
        \csitem{Vars, Functions, Methods}{Lowercase, seperate by underscore}
        \csitem{Constants}{All uppercase}
    \end{cschapter}

    %%%%%%%%%%%%%%%%%%%%%%%%%%%%%%%%%%%%%%%%
    \begin{cschapter}{\href{http://docs.python.org/3/library/stdtypes.html}{String Methods}}
        \csitem{\code{find(sub[,start[,end]])}}{Lowest index of substring 'sub' in slice s[start,end].}
        \csitem{\code{join(iterable)}}{Concatenate items in iterable with seperater 's'.}
        \csitem{\code{replace(old, new[, count])}}{Replace 'old' with 'new' in string.}
        \csitem{\code{split(sep[,maxsplit])}}{List of words, using 'sep' as delimiter string.}
        \csitemhead{\textbf{printf-style formatting}}
        \csitemhead{\code{"\%i \%i \%i" \% (42, 43, 44) --> "42 43 44"}}
        \csitem{\code{'i'}}{Signed integer decimal.}
        \csitem{\code{'x', 'X'}}{Signed hexadecimal (lowercase, uppercase).}
        \csitem{\code{'e', 'f'}}{Float exponential, decimal.}
        \csitem{\code{'r'}}{String (repr()).}
        \csitem{\code{'s'}}{String (str()).}
    \end{cschapter}

    %%%%%%%%%%%%%%%%%%%%%%%%%%%%%%%%%%%%%%%%
    \begin{cschapter}{Generator}
        \csitemhead{Creates iterator ('list') dynamically without using return.}
        \csitem{\code{def f(i):}}{Generator function.}
        \csitem{\code{{\space\space\space\space}for j in range(i):}}{}
        \csitem{\code{{\space\space\space\space}{\space\space\space\space}yield j*2}}{Yield a value to iterator.}
        \csitem{\code{for g in f(42):}}{Call Generator function.}
        \csitem{\code{{\space\space\space\space}print(g)}}{Do something with values.}
    \end{cschapter}

    %%%%%%%%%%%%%%%%%%%%%%%%%%%%%%%%%%%%%%%%
    \begin{cschapter}{\href{http://docs.python.org/3/library/functions.html}{Built-in Functions}}
        \csitem{\code{abs(x)}}{Absolute value of integer.}
        \csitem{\code{bin(x)}}{Binary string of integer.}
        \csitem{\code{filter(f,lst)}}{Apply function 'f' to all in 'lst'.}
        \csitem{\code{hash(object)}}{Get hash for object.}
        \csitem{\code{hex(x)}}{Hex string for integer.}
        \csitem{\code{isinstance(o,cls)}}{True if 'o' is instance of 'cls' or sublcass.}
        \csitem{\code{len(s)}}{Length of object (string, tuple, list, dict).}
        \csitem{\code{print(objects...)}}{Print objects to stdout.}
        \csitem{\code{range(stop)}}{Returns list of integers 0 to stop.}
        \csitem{\code{reversed(seq)}}{Returns reversed list.}
        \csitem{\code{sorted(sorted)}}{Returns sorted list.}
        \csitem{\code{type(object)}}{Returns type of object.}
    \end{cschapter}

    %%%%%%%%%%%%%%%%%%%%%%%%%%%%%%%%%%%%%%%%
    \begin{cschapter}{File handling}
        \csitem{\code{fd = open(file, mode)}}{Open a file handler.}
        \csitem{\code{fd.close()}}{Close a file handler.}
        \csitemhead{\textbf{Modes}}
        \csitem{r}{Reading (default).}
        \csitem{w}{Writing, truncating the file first.}
        \csitem{x}{Exclusive creation, failing if the file already exists.}
        \csitem{a}{Appending to the end of the file.}
        \csitem{rb}{Reading binary mode. Don't use for text files!}
        \csitem{wb}{Writing binary mode. Don't use for text files!}
    \end{cschapter}

    %%%%%%%%%%%%%%%%%%%%%%%%%%%%%%%%%%%%%%%%
    \begin{cschapter}{Import}
        \csitem{\code{import X}}{Import module X into namespace.}
        \csitem{\code{from X import *}}{Import public content of module X into namespace. Avoid this!}
        \csitem{\code{from X import a, b, c}}{Import objects from module X into namespace.}
        \csitem{\code{X = \_\_import\_\_('X')}}{Equal to \code{import X}, but imports into variable.}
    \end{cschapter}

    %%%%%%%%%%%%%%%%%%%%%%%%%%%%%%%%%%%%%%%%
    \begin{cschapter}{\href{http://docs.python.org/3/library/math.html}{math}}
        \csitem{\code{ceil(x)}}{Ceiling of x.}
        \csitem{\code{fabs(x)}}{Absolute value of x.}
        \csitem{\code{factorial(x)}}{X factorial.}
        \csitem{\code{floor(x)}}{Floor value of x.}
        \csitem{\code{exp(x)}}{Returns e**x.}
        \csitem{\code{log(x[,base=e])}}{Logarithm of x.}
        \csitem{\code{pow(x,y)}}{x raised to the power y.}
        \csitem{\code{sqrt(x)}}{Square root of x.}
        \csitem{\code{sin(x), cos(x), tan(x)}}{Sine, cosine, tagent of x.}
    \end{cschapter}

    %%%%%%%%%%%%%%%%%%%%%%%%%%%%%%%%%%%%%%%%
    \begin{cschapter}{\href{http://docs.python.org/3/library/sys.html}{sys}}
        \csitem{\code{argv}}{List of command line arguments.}
        \csitem{\code{exit([arg])}}{Exit from python using 'arg' as status.}
        \csitem{\code{getsizeof(object)}}{Return size of object in bytes.}
        \csitem{\code{path}}{Search path for modules.}
        \csitem{\code{platform}}{Platform identifier.}
        \csitemhead{\textbf{Platforms}}
        \csitem{Linux}{'linux'}
        \csitem{Windows}{'win32'}
        \csitem{Mac OS X}{'darwin'}
    \end{cschapter}

    %%%%%%%%%%%%%%%%%%%%%%%%%%%%%%%%%%%%%%%%
    \begin{cschapter}{\href{http://docs.python.org/3/library/os.html}{os}}
        \csitem{name}{Name of OS dependent module imported.}
        \csitemhead{Names: 'posix', 'nt', 'mac', 'os2', 'ce', 'java'.}
        \csitem{\code{getenv(key)}}{Get enviroment variable.}
        \csitem{\code{listdir(path='.')}}{List of entries in the dir.}
        \csitem{\code{curdir}}{Current dir constant ('.').}
        \csitem{\code{pardir}}{Parent dir constant ('..').}
        \csitem{\code{sep}}{Pathname seperator ('/').}
    \end{cschapter}

    %%%%%%%%%%%%%%%%%%%%%%%%%%%%%%%%%%%%%%%%
    % TODO: 
    \begin{cschapter}{datetime}
        \csitem{}{}
    \end{cschapter}

    %%%%%%%%%%%%%%%%%%%%%%%%%%%%%%%%%%%%%%%%
    % TODO: argparse
    \begin{cschapter}{argparse}
        \csitem{}{}
    \end{cschapter}

    %%%%%%%%%%%%%%%%%%%%%%%%%%%%%%%%%%%%%%%%
    % TODO: 
    \begin{cschapter}{ctypes}
        \csitem{}{}
    \end{cschapter}

    %%%%%%%%%%%%%%%%%%%%%%%%%%%%%%%%%%%%%%%%
    % TODO: 
    \begin{cschapter}{random}
        \csitem{}{}
    \end{cschapter}

    %%%%%%%%%%%%%%%%%%%%%%%%%%%%%%%%%%%%%%%%
    % TODO: 
    \begin{cschapter}{re}
        \csitem{}{}
    \end{cschapter}

    %%%%%%%%%%%%%%%%%%%%%%%%%%%%%%%%%%%%%%%%
    % TODO: 
    \begin{cschapter}{urllib}
        \csitem{}{}
    \end{cschapter}

    %%%%%%%%%%%%%%%%%%%%%%%%%%%%%%%%%%%%%%%%
    % TODO: 
    \begin{cschapter}{json}
        \csitem{}{}
    \end{cschapter}

    %%%%%%%%%%%%%%%%%%%%%%%%%%%%%%%%%%%%%%%%
    % TODO: 
    \begin{cschapter}{socket}
        \csitem{}{}
    \end{cschapter}

    %%%%%%%%%%%%%%%%%%%%%%%%%%%%%%%%%%%%%%%%
    % TODO: 
    \begin{cschapter}{threading}
        \csitem{}{}
    \end{cschapter}


    % License
    \rule{0.3\linewidth}{0.25pt}
    \scriptsize

    Written for and tested with Python 3.3.1

    Copyright \copyright\ 2013 Martin Wichmann

    Licensed under a CC-BY-SA License

\end{multicols}
\end{document}



